\documentclass[11pt,a4paper,reqno]{article}
\linespread{1.5}

\usepackage[active]{srcltx}
\usepackage{listings}
\usepackage{graphicx}
\usepackage{amsthm,amsfonts,amsmath,amssymb,indentfirst,mathrsfs,amscd}
\usepackage[mathscr]{eucal}
\usepackage[active]{srcltx} %inverse search
\usepackage{tensor}
\usepackage[utf8x]{inputenc}
\usepackage[portuges]{babel}
\usepackage[T1]{fontenc}
\usepackage{enumitem}
\setlist{nolistsep}
\usepackage{comment} 
\usepackage{tikz}
\usepackage[numbers,square, comma, sort&compress]{natbib}
\usepackage[nottoc,numbib]{tocbibind}
%\numberwithin{figure}{section}
\numberwithin{equation}{section}
\usepackage{scalefnt}
\usepackage[top=2.5cm, bottom=2.5cm, left=2.5cm, right=2.5cm]{geometry}
%\usepackage{tweaklist}
%\renewcommand{\itemhook}{\setlength{\topsep}{0pt}%
%	\setlength{\itemsep}{0pt}}
%\renewcommand{\enumhook}{\setlength{\topsep}{0pt}%
%	\setlength{\itemsep}{0pt}}
%\usepackage[colorlinks]{hyperref}
\usepackage{MnSymbol}
%\usepackage[pdfpagelabels,pagebackref,hypertexnames=true,plainpages=false,naturalnames]{hyperref}
\usepackage[naturalnames]{hyperref}
\usepackage{enumitem}
\usepackage{titling}
\newcommand{\subtitle}[1]{%
	\posttitle{%
	\par\end{center}
	\begin{center}\large#1\end{center}
	\vskip0.5em}%
}
\newcommand{\HRule}{\rule{\linewidth}{0.5mm}}
\usepackage{caption}
\usepackage{etoolbox}% http://ctan.org/pkg/etoolbox
\usepackage{complexity}

\usepackage[official]{eurosym}

\def\Cpp{C\raisebox{0.5ex}{\tiny\textbf{++}}}

\makeatletter
\def\@makechapterhead#1{%
  %%%%\vspace*{50\p@}% %%% removed!
  {\parindent \z@ \raggedright \normalfont
    \ifnum \c@secnumdepth >\m@ne
        \huge\bfseries \@chapapp\space \thechapter
        \par\nobreak
        \vskip 20\p@
    \fi
    \interlinepenalty\@M
    \Huge \bfseries #1\par\nobreak
    \vskip 40\p@
  }}
\def\@makeschapterhead#1{%
  %%%%%\vspace*{50\p@}% %%% removed!
  {\parindent \z@ \raggedright
    \normalfont
    \interlinepenalty\@M
    \Huge \bfseries  #1\par\nobreak
    \vskip 40\p@
  }}
\makeatother

\usepackage[toc,page]{appendix}

\addto\captionsportuges{%
  \renewcommand\appendixname{Anexo}
  \renewcommand\appendixpagename{Anexos}
}

\addto\captionsportuges{%
  \renewcommand\abstractname{Sumário}
}

\begin{document}



\begin{titlepage}
\begin{center}

{\large 3º ano do Mestrado Integrado em Engenharia Informática e Computação \\[5mm]}
{\Large Inteligência Artificial}\\[2cm]

% Title
{\Huge \bfseries Otimização da Localização de Tribunais em Portugal \\[5mm]}
{\large \textit{Relatório Intecalar} \\[2cm]}

\includegraphics[width=12.5cm]{feup_logo.jpg}\\[2cm]

% Author
{\large Flávio Couto - up201303726 - up201303726@fe.up.pt\\[1mm]}
{\large Luís Figueiredo - up201304295 - up201304295@fe.up.pt \\[1mm]}
{\large Pedro Afonso Castro - up201304205 - up201304205@fe.up.pt\\[1mm]}


% Bottom of the page
{\large \today}

\end{center}
\end{titlepage}

\tableofcontents

%%%%%%%%%%%%%%
%  OBJETIVO  %
%%%%%%%%%%%%%%
\section{Objetivo}

O objetivo do trabalho é distribuir de uma maneira eficiente um número fixo de tribunais pelos concelhos do país, sendo o número de tribunais substancialmente inferior ao de concelhos, recorrendo a métodos de pesquisa de soluções para problemas de otimização como algoritmos genéticos e arrefecimento simulado. 


%%%%%%%%%%%%%
% DESCRIÇÃO %
%%%%%%%%%%%%%
\section{Descrição}
\subsection{Especificação}

Dada a natureza dos algoritmos de otimização, é necessário encontrar uma função de avaliação que nos permita determinar a qualidade de uma solução. Para determinar a distribuição ideal dos tribunais, temos então de determinar os fatores que nos permitam considerar uma solução como ótima. Posto isto, podemos enumerar os seguintes fatores:

\begin{itemize}
\item O número de cidadãos residentes em concelhos onde sejam colocados tribunais deve ser o maior possível;
\item A distância a um tribunal dos cidadãos residentes em concelhos onde não exista um tribunal deve ser a menor possível;
\item Nenhum cidadão deverá ficar a uma distância superior a X de um tribunal.
\end{itemize}

No entanto, encontrar estes fatores não é suficiente. Precisamos de uma forma de representar cada possível solução. O grupo optou por usar uma representação baseada numa cadeia binária (de 0's e 1's), com tamanho igual ao número de concelhos, em que um 0 na posição i representa a inexistência de um tribunal no concelho i, e um 1 representa a existência de um tribunal nesse concelho.

Estamos então em condições de determinar uma função de avaliação para o problema em questão. A seguinte função de avaliação será utilizada:

f(x) = $\sum\limits_{i=1}^{n}\
	\begin{cases} 
		nCidadaos[i] \mbox{ se } s[i] = 1\\
		\begin{cases}
		min(dist(i,x)) \mbox{ em que }  $\{ $x$ \mid x \in cities, s[x] = 1 \}$ \mbox{ se } s[i] = 0\\
		0 \mbox{ se } max(dist(i,x)) > maxDist \mbox{ em que }$ \{ $x$ \mid x \in cities, s[x] = 1 \}$
		\end{cases}
	\end{cases}
	
Em que nCidadaos representa uma lista com os cidadãos das cidades, s representa a cadeira referida anteriormente, cities representa uma lista de todas as cidades e maxDist a distância máxima a que um cidadão pode ficar de um tribunal.
\newline

O grupo resolveu dividir o trabalho nas seguintes fases:
\begin{itemize}
\item Extracção do nome, população e coordenadas geográficas de cada sede de concelho em Portugal;
\item Implementação genérica de um algoritmo genético (ou seja, uma implementação que se consiga aplicar a qualquer problema que possa ser resolvido recorrendo a este algoritmo);
\item Implementação genérica de arrefecimento simulado;
\item Determinação da função de avaliação e forma de representação a ser utilizada nos algoritmos de optimização implementados;
\item Criação de uma interface gráfica que permita ao utilizador encontrar uma solução para o problema de forma simples e intuitiva;
\item Implementação de melhorias não especificadas no enunciado que o grupo entenda, caso haja disponibilidade (como por exemplo, a utilização de outros algoritmos de otimização).
\end{itemize}


\subsection{Trabalho realizado}
Na altura da entrega deste relatório, o grupo já extraiu informações realistas do nome, população e localização de cada sede de concelho em Portugal.

O grupo também já terminou a implementação genérica de um algoritmo genético, bem como a determinação da função de avaliação e forma de representação de cada possível solução.


%%%%%%%%%%%%%%%%
% BIBLIOGRAPHY %
%%%%%%%%%%%%%%%%
\bibliographystyle{IEEEtran}
\bibliography{rabb,refs}

\end{document}
