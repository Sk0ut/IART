\documentclass[11pt,a4paper,reqno]{article}
\linespread{1.5}

\usepackage[active]{srcltx}
\usepackage{listings}
\usepackage{graphicx}
\usepackage{amsthm,amsfonts,amsmath,amssymb,indentfirst,mathrsfs,amscd}
\usepackage[mathscr]{eucal}
\usepackage[active]{srcltx} %inverse search
\usepackage{tensor}
\usepackage[utf8x]{inputenc}
\usepackage[portuges]{babel}
\usepackage[T1]{fontenc}
\usepackage{enumitem}
\setlist{nolistsep}
\usepackage{comment} 
\usepackage{tikz}
\usepackage[numbers,square, comma, sort&compress]{natbib}
\usepackage[nottoc,numbib]{tocbibind}
%\numberwithin{figure}{section}
\numberwithin{equation}{section}
\usepackage{scalefnt}
\usepackage[top=2.5cm, bottom=2.5cm, left=2.5cm, right=2.5cm]{geometry}
%\usepackage{tweaklist}
%\renewcommand{\itemhook}{\setlength{\topsep}{0pt}%
%	\setlength{\itemsep}{0pt}}
%\renewcommand{\enumhook}{\setlength{\topsep}{0pt}%
%	\setlength{\itemsep}{0pt}}
%\usepackage[colorlinks]{hyperref}
\usepackage{MnSymbol}
%\usepackage[pdfpagelabels,pagebackref,hypertexnames=true,plainpages=false,naturalnames]{hyperref}
\usepackage[naturalnames]{hyperref}
\usepackage{enumitem}
\usepackage{titling}
\newcommand{\subtitle}[1]{%
	\posttitle{%
	\par\end{center}
	\begin{center}\large#1\end{center}
	\vskip0.5em}%
}
\newcommand{\HRule}{\rule{\linewidth}{0.5mm}}
\usepackage{caption}
\usepackage{etoolbox}% http://ctan.org/pkg/etoolbox
\usepackage{complexity}

\usepackage[official]{eurosym}

\def\Cpp{C\raisebox{0.5ex}{\tiny\textbf{++}}}

\makeatletter
\def\@makechapterhead#1{%
  %%%%\vspace*{50\p@}% %%% removed!
  {\parindent \z@ \raggedright \normalfont
    \ifnum \c@secnumdepth >\m@ne
        \huge\bfseries \@chapapp\space \thechapter
        \par\nobreak
        \vskip 20\p@
    \fi
    \interlinepenalty\@M
    \Huge \bfseries #1\par\nobreak
    \vskip 40\p@
  }}
\def\@makeschapterhead#1{%
  %%%%%\vspace*{50\p@}% %%% removed!
  {\parindent \z@ \raggedright
    \normalfont
    \interlinepenalty\@M
    \Huge \bfseries  #1\par\nobreak
    \vskip 40\p@
  }}
\makeatother

\usepackage[toc,page]{appendix}

\addto\captionsportuges{%
  \renewcommand\appendixname{Anexo}
  \renewcommand\appendixpagename{Anexos}
}

\addto\captionsportuges{%
  \renewcommand\abstractname{Sumário}
}

\begin{document}



\begin{titlepage}
\begin{center}

{\large 3º ano do Mestrado Integrado em Engenharia Informática e Computação \\[5mm]}
{\Large Inteligência Artificial}\\[3cm]

\includegraphics[width=12.5cm]{feup_logo.jpg}\\[3cm]

% Title
{\Huge \bfseries Otimização da Localização de Tribunais em Portugal \\[5mm]}
{\large \textit{Relatório Intecalar} \\[2cm]}


% Author
{\large Flávio Couto - up201303726 - up201303726@fe.up.pt\\[1mm]}
{\large Luís Figueiredo - up201304295 - up201304295@fe.up.pt \\[1mm]}
{\large Pedro Afonso Castro - up201304205 - up201304205@fe.up.pt\\[1cm]}


% Bottom of the page
{\large \today}

\end{center}
\end{titlepage}

\tableofcontents


%%%%%%%%%%%%%%
%  OBJETIVO  %
%%%%%%%%%%%%%%

\newpage

\section{Objetivo}

Este projeto encpntra-se a ser desenvolvido no âmbito da unidade curricular de Inteligência Artificial (IART) e tem como principal objetivo realizar um estudo sobre algoritmos de otimização de soluções. Mais concretamente será abordada a otimização do mapa judiciário português, isto é, a distribuição de um número fixo de tribunais pelos concelhos do país. Sendo o número de tribunais substancialmente inferior ao de concelhos, recorreremos a metodologias de otimização de soluções como algoritmos genéticos e arrefecimento simulado para atingirmos ao objetivo. 


%%%%%%%%%%%%%
% DESCRIÇÃO %
%%%%%%%%%%%%%

\newpage

\section{Descrição} 
\subsection{Especificação} \label{introduction}

Dada a natureza dos algoritmos de otimização, é necessário encontrar uma função de avaliação que nos permita determinar a qualidade de uma solução. Para determinar a distribuição ideal dos tribunais, temos então de determinar os fatores que nos permitam considerar uma solução como ótima. Posto isto, podemos enumerar os seguintes fatores:

\begin{itemize}
\item O número de cidadãos residentes em concelhos onde sejam colocados tribunais deve ser o maior possível;
\item A distância a um tribunal dos cidadãos residentes em concelhos onde não exista um tribunal deve ser a menor possível;
\item Nenhum cidadão deverá ficar a uma distância superior a X de um tribunal.
\item Os custos de construção dos tribunais deverão ser os menores possíveis.
\end{itemize}

No entanto, encontrar estes fatores não é suficiente. Precisamos de uma forma de representar cada possível solução. O grupo optou por usar uma representação baseada numa cadeia binária (de 0's e 1's), com tamanho igual ao número de concelhos, em que um 0 na posição i representa a inexistência de um tribunal no concelho i, e um 1 representa a existência de um tribunal nesse concelho.

Estamos então em condições de determinar uma função de avaliação para o problema em questão. A seguinte função de avaliação será utilizada:

\begin{equation}
    H(X_i) = \begin{cases}
    -\infty ,\ se \sum\limits_{j}X_i[j]\ > T\textsubscript{max} \\
    \sum\limits_{j}\ h(X_i, j]), \ se \sum\limits_{j}\ X_i[j] <= T\textsubscript{max}\\
    \end{cases}
\end{equation}

\begin{equation}    
h(X_i, j) = \begin{cases} 
    P[j] * V_p - C[j] ,\  se \  X_i[j] = 1 \\
    P[j] * V_p * \frac{D\textsubscript{max} - min(D(j, k))}{D\textsubscript{max}}, \ se \ X_i[j] = 0 \ \wedge k \in L(j, D\textsubscript{max}) \ \wedge X_i[k] = 1 \\
    -P[j] * C_p,\  se \ X_i[j] = 0 \ \wedge \ \forall _k \ (k \in L(j, D\textsubscript{max}) \ \implies \  X_i[k] = 0) \\
	\end{cases}\\
\end{equation}
	
\	
Em que:
\begin{itemize}
\item $X_i$ representa o cromossoma número $i$ da geração atual;
\item $T\textsubscript{max}$ representa o número máximo de tribunais a distribuir
\item $P[j]$ representa o número de pessoas da cidade $j$;
\item $V_p$ representa o valor de uma pessoa ter acesso a um tribunal perto dela;
\item $C_p$ representa o custo de uma pessoa não ter acesso a um tribunal perto dela;
\item $C_t[j]$ representa o custo de construir um tribunal na cidade $j$;
\item $X_i[j]$ representa a existência (1) ou não (0) de um tribunal na cidade $j$;
\item $D\textsubscript{max}$ representa a distância máxima desejada de uma cidade a um tribunal;
\item $D(j, k)$ representa a distância entre as cidades $j$ e $k$;
\item $L(j, D\textsubscript{max})$ representa o conjunto das cidades que se encontram no máximo a $D\textsubscript{max}$ da cidade $j$, excluindo $j$;
\end{itemize} 

\ 
Para tornar possível a seleção de cromossomas para emparelhamento, serão consideradas diferentes adaptações da função de avaliação acima apresentada, cada uma com diferentes vantagens e desvantagens:

\begin{equation} % Problema: soluções válidas mas negativas serão descartadas, possibilidade de matança de todos os cromossomas
    H1(X_i) = max(0, H(X_i))
\end{equation}

\
Esta função de avaliação descarta as soluções com mais tribunais que os permitidos e as soluções com avaliação negativa, apesar de respeitar o número máximo de tribunais.

\begin{equation} % Problema: não há certezas em relação à escala da compensação: a diferença entre o pior cromossoma e o melhor pode ser minuscula
    H2(X_i) = max(0, H(X_i) - \sum\limits_{j} min(P[j] * V_p - C[j], -P[j] * C_p))
\end{equation}

\
Esta função de avaliação apenas descarta as soluções com mais tribunais que os permitidos e a solução com o menor valor de avaliação possível de ser obtido, incrementando todos os valores da função de avaliação $H(X_i)$. No entanto, esta função pode dificultar a diferenciação entre soluções, caso a parcela de compensação tenha uma contribuição demasiado grande.

\begin{equation}
    H3(X_i) = max(0, H(X_i) - min(0, min(\forall _j  (H(X_j)))))
\end{equation}

\
Esta função de avaliação descarta as soluções com mais tribunais que os permitidos e a pior solução da geração atual. Isto permite reduzir o problema da contribuição apresentado para a função $H2(X_i)$, mas introduz um \textit{overhead} para determinação do menor valor de avaliação da geração atual.

De forma a tornar esta aplicação o mais realista possível, consideramos desenvolver um módulo que obtenha dados relativos às cidades e sua distribuição pelo país através de fontes fiáveis, nomeadamente a \textit{API} da \textit{Google Maps}. 

O grupo resolveu dividir o trabalho nas seguintes fases:
\begin{itemize}
\item Extracção do nome, população e coordenadas geográficas de cada sede de concelho em Portugal;
\item Implementação genérica de um algoritmo genético (ou seja, uma implementação que se consiga aplicar a qualquer problema que possa ser resolvido recorrendo a este algoritmo);
\item Implementação genérica de arrefecimento simulado;
\item Determinação da função de avaliação e forma de representação a ser utilizada nos algoritmos de optimização implementados;
\item Criação de uma interface gráfica que permita ao utilizador encontrar uma solução para o problema de forma simples e intuitiva;
\item Implementação de melhorias não especificadas no enunciado que o grupo entenda, caso haja disponibilidade (como por exemplo, a utilização de outros algoritmos de otimização).
\end{itemize}


\subsection{Trabalho realizado}
Na altura da entrega deste relatório, o grupo já extraiu informações realistas do nome, população e localização de cada sede de concelho em Portugal.

O grupo também já terminou a implementação genérica de um algoritmo genético, bem como a determinação da função de avaliação e forma de representação de cada possível solução.

\subsection{Resultados esperados e forma de avaliação}
Para avaliar a qualidade dos algoritmos implementados, podemos focar-nos em dois aspetos: qualidade da solução e tempo de computação.

Como já foi especificado anteriormente, para avaliar a qualidade de uma solução, a função heurística \textit{h(n)} especificada anteriormente será utilizada. Esta função será também usada para avaliar a qualidade da solução final gerada pelo algoritmo utilizado, permitindo a sua validação.

Com isto, podemos avaliar a qualidade das soluções geradas pelos algoritmos, podendo assim ver qual é o mais eficiente. Gerando uma amostra estatisticamente representativa (30 soluções) para cada algoritmo e calculando a média do resultado da função de avaliação para cada amostra, poderemos assim ver qual dos algoritmos é mais eficaz para a resolução do problema.

Para calcular o tempo de execução, basta determinar o tempo imediatamente antes e depois da execução do algoritmo, permitindo-nos assim saber qual dos algoritmos é que permite encontrar uma solução mais rapidamente.


%%%%%%%%%%%%%
% CONCLUSÃO %
%%%%%%%%%%%%%

\newpage

\section{Conclusão}
Após reflexão entre os membros do grupo acerca do tema escolhido, não só quando começámos a trabalhar neste, como também na produção deste relatório intercalar, chegámos à conclusão que fizemos uma boa escolha do tema do projeto. É um projeto bastante interessante que, sendo desafiante, não é demasiado trabalhoso.

Pensar na forma mais correta de implementar os algoritmos, bem como encontrar a função de avaliação mais adequada foram os aspetos mais interessantes do desenvolvimento do projeto até agora, e estamos motivados para fazer o melhor trabalho possível nas semanas que faltam.

Está a ser definitivamente um dos trabalhos mais cativantes do semestre, e esperamos gostar tanto de trabalhar nele a partir de agora como gostámos até aqui.

%%%%%%%%%%%%%%%%
% BIBLIOGRAPHY %
%%%%%%%%%%%%%%%%

\newpage

\section{Recursos utilizados}

\begin{itemize}
\item Apontamentos das aulas
\item MILLER, Brad L. and David E. Goldberg. \textit{Genetic Algorithms, Tournament Selection, and the Effects of Noise}, University of Illinois, 1995\item Creating a genetic algorithm for beginners (http://www.theprojectspot.com/tutorial-post/creating-a-genetic-algorithm-for-beginners/3)
\item IntelliJ IDEA (https://www.jetbrains.com/idea/)
\item TeXmaker (http://www.xm1math.net/texmaker/)
\item LaTeX(https://www.latex-project.org/)
\item Git (https://git-scm.com/)

\end{itemize}

\end{document}
